\documentclass[italian,12pt,a4paper,oneside,final]{report}
%\documentclass[italian,a4paper,titlepage]{article}
%\usepackage{amsmath}
%\usepackage{caption}
%\usepackage{subcaption}
\usepackage{graphicx}
\usepackage{biblatex} %Imports biblatex package
\usepackage[utf8]{inputenc}
\usepackage[italian]{babel}
\usepackage{csquotes}
%\usepackage[T1]{fontenc}
%\usepackage{longtable}
%\usepackage{booktabs}
%\usepackage{textcomp}
%\usepackage[draft=false]{hyperref}
%\usepackage{hyperref}
%\usepackage[table]{xcolor}
\addbibresource{iot.bib} %Import the bibliography file
\graphicspath{ {images/} }
\renewcommand{\thesection}{\arabic{section}} % remove the \chapter counter from being printed with every \section
%\hypersetup{
%	colorlinks=true,
%	linkcolor=,
%	pdftitle={Marco Giunta - Progetto IoT},
%	pdfauthor={Marco Giunta},
%}

\title{\huge Power Meter Wi-Fi con Arduino\\[0.5em]
\large Relazione Progetto IoT}
\date{Ottobre 2022}
\author{
Marco Giunta\thanks{Marco Giunta 147852 giunta.marco@spes.uniud.it}}

\begin{document}
% Generate title page
\maketitle

% Generate TOCs
\pagenumbering{arabic}
\tableofcontents

\newpage

\section{Introduzione}
In questo periodo storico, dove i problemi causati dal riscaldamento globale sono sotto gli occhi di tutti, si rende necessario ridurre il consumo energetico all'interno delle nostre case o negli edifici pubblici.
Per poter decidere quali azioni intraprendere, bisogna prima conoscere il reale consumo delle nostre apparecchiature.
Con questo progetto si vuole realizzare un prototipo a basso costo di un misuratore di consumo elettrico collegato ad una rete locale tramite Wi-Fi.
Il progetto ha come obiettivo l’acquisizione e il monitoraggio dei valori di tensione, corrente e potenza presenti ai capi di un qualunque apparecchio alimentato a 220V.
Il codice presente all'interno del dispositivo è stato progettato per permettere il collegamento contemporaneo di circa 65000 unità, in modo da poter monitorare, ad esempio, tutte le apparecchiature presenti all'interno di un istituto di ricerca.
Per garantire il monitoraggio di un numero così elevato di dispositivi, è stato necessario ricorrere all'utilizzo del protocollo MQTT\footfullcite{mqtt}, per ottimizzare la gestione della banda di rete e autenticare i singoli dispositi.



La parte del monitoraggio si appoggia a due servizi cloud gratuiti, InfluxDB per
la creazione ed il mantenimento del database e Grafana per la parte di visualiz-
zazione tramite dashboard.
Nonostante InfluxDB proponga un servizio di visualizzazione con dashboard,
viene scelto Grafana per una migliore estetica.
In caso serva un database completo, i dati rilevati dall’Arduino Uno vengo sal-
vati su una scheda SD in modo da prevenire disconnessioni e garantire una facile
futura elaborazione dei dati senza l’utilizzo dei servizi cloud.

\section{Sensori}



Lo scopo di questa settima esperienza di laboratorio è determinare i valori delle costanti di taratura della funzione di trasferimento di un sensore di temperatura, usato come trasduttore tra le grandezze temperatura e resistenza.
Misureremo la temperatura di un bagno termico attraverso un termometro ad alcool e una serie di termistori NTC.
Inoltre determineremo il tempo di risposta di uno dei termistori NTC utilizzati durante l'esperimento.

\subsection{Trasduzione di grandezze fisiche}
In tutti i casi in cui si vuole misurare una grandezza fisica, ma non è possibile confrontarla con uno o più campioni o definirla mediante una misura indiretta, si ricorre alla \emph{correlazione} della grandezza fisica con un’altra, più semplice da misurare.
I dispositivi che permettono questo tipo di correlazione (\emph{trasduzione}) vengono chiamati \textbf{trasduttori}.
\end{document}